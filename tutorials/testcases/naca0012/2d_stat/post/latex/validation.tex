\documentclass[a4paper,10pt]{scrartcl}
\usepackage[utf8]{inputenc}

\usepackage{amsmath}
\usepackage{booktabs}
\usepackage{graphicx}
\usepackage[margin=15mm]{geometry}

\footskip=20pt


%opening
\title{Validation of Bellerophon NACA 0012 wing in 2D flow}
\author{Andreas Groß}

\begin{document}

\maketitle

\section{Case Description}
This test case compares the results for a steady turbulent flow over a NACA 0012
profile at a range of angles of attack. 

The Reynolds number of this test case is \(2.88\times10^6\). For turbulence modeling
the \(k\text{-}\epsilon\)-model is used\cite{kEps}. Different mesh sizes are used for
validation of the methods used. \textbf{TODO}

Results are compared against the validation results provided by Ladson\cite{Ladson1}, Ladson et al.\cite{Ladson2} and
Gregory et al.\cite{Gregory}.

\section{Results}

The lift coefficient for angles of attack between \(0^\circ\) and \(20^\circ\) are
shown in Figure~\ref{Lift}. There is a good agreement for lift at low and moderate angles
of attack. For high angles of attack, the calculation predict a lower lift then the experiments.
The calculated drag is shown in Figure~\ref{Lift} and is higher then the drag measured.
\begin{figure}[!htbp]
  \centering
  \resizebox{0.49\columnwidth}{!}{\input{../results/CLoverAlpha.tex}}
  \resizebox{0.49\columnwidth}{!}{\input{../results/CDoverCL.tex}}
  \caption{Lift coefficient over \(\alpha\) (left) and Drag vs. Lift (right)}
  \label{Lift}
\end{figure}

A comparison of the pressure distribution at angles of attack of \(0^\circ\) and \(10^\circ\) is
shown in Figures~\ref{CP}. The results for \(0^\circ\) show good agreement with the experimental
results. For \(10^\circ\) the experimental results differ in the pressure peak at the upper
surface\cite{Gregory}\cite{Ladson2}. The results by Gergory et al. seem to resolve this peak
better as they might be considered to be more two dimensional. The calculations show good agreement
with these data.

\begin{figure}[!htbp]
  \centering
  \resizebox{0.49\columnwidth}{!}{\input{../results/pressure_0deg.tex}}
  \resizebox{0.49\columnwidth}{!}{\input{../results/pressure_10deg.tex}} \\
  \resizebox{0.49\columnwidth}{!}{\input{../results/pressure_15deg.tex}} \\
  \caption{Pressure distribution at \(0^\circ\) (left) and \(10^\circ\) (right)}
  \label{CP}
\end{figure}

\bibliography{validation}
\bibliographystyle{unsrt}

\end{document}
