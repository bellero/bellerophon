\documentclass[a4paper,10pt]{scrartcl}
\usepackage[utf8]{inputenc}

\usepackage{amsmath}
\usepackage{booktabs}
\usepackage{graphicx}
\usepackage[margin=15mm]{geometry}


%opening
\title{Validation of Bellerophon for Transient Cylinder Flow}
\author{Andreas Groß}

\begin{document}

\maketitle

\section{Case Description}
This testcase compares the results for a transient two dimensional flow around
a round cylinder with \(Re=10^5\). The diameter of the cylinder is
\(D=1\ \text m\) and the inlet velocity is set to
\(U=1\frac{\text m}{\text s}.\). The Strouhal number for this case should be
around 0.19.

The flow is considered as laminar, which might not be suitable for a correct
representation of the physical problem. This testcase however only compares the
results of the overset grid with a single grid and therefore is still
usable.

The distance of the boundaries from the cylinder is given in
Table~\ref{domain}. The results for both cases are compared after 100 seconds
(\(\sim\) 20 oscilations) for another 100 seconds.

\begin{table}[!htbp]
 \caption{Dimensions of the Domain}
 \centering
 \begin{tabular}{lc}
  \toprule
  Direction & Dimension \\
  \midrule
  Upstream & 20 m\\
  Sides & 20 m\\
  Downstream & 20 m\\
  Diameter of overset region & 12 m \\
  Diameter of hole in background mesh & 9.2 m \\
  \bottomrule
  \label{domain}
 \end{tabular}
\end{table}


\section{Results}

The time resolved results for drag and lift of the cylinder are shown in
Figure~\ref{cd_timedomain}~and~\ref{cl_timedomain}. The time averaged values
are summarized in Table \ref{results}. There is a good agreement of the time
averaged values of drag and lift coefficient. The standard deviations of the
values is below 0.5 \% showing good agreement of the oscilation amplitudes of
both drag and lift. The averaged lift should be zero. Both results show lift
close to zero with small deviations. Due to the small magnitude of the value,
the relative difference is high and should be neglected.

%\begin{figure}[!htbp]
%  \centering
%  \input{../results/cd_timedomain.tex}
%  \caption{Spectrum of Drag Coefficient}
%  \label{cd_timedomain}
%\end{figure}

%\begin{figure}[!htbp]
%  \centering
%  \input{../results/cl_timedomain.tex}
%  \caption{Spectrum of Lift Coefficient}
%  \label{cl_timedomain}
%\end{figure}

%\begin{table}[!htbp]
% \caption{Results for Lift and Drag}
% \centering
% \begin{tabular}{lccc}
\toprule
 & Overset & Single & Deviation \\
\toprule
Avg. Drag &1.3277 &1.3438 &-1.2011 \% \\
SD Drag &0.14264 &0.14436 &-1.1914 \% \\
Avg. Lift &0.005138 &0.0033593 &52.95 \% \\
SD Lift &1.3449 &1.3562 &-0.83373 \% \\
\toprule
\end{tabular}

% \label{results}
%\end{table}

The results for lift and drag in frequency domain are shown in
Figure~\ref{cd_timedomain}~and~\ref{cl_spectra}. The shedding frequency of the
vortex street shows good agreement for the overset and the single grid. The
comparison to the experimental values for this case is bad. This may be
improved by a increased mesh resolution or a suitable turbulence model.

\begin{figure}[!htbp]
  \centering
  \input{../results/cl_spectra.tex}
  \caption{Spectrum of Drag Coefficient}
  \label{cd_spectra}
\end{figure}

\begin{figure}[!htbp]
  \centering
  \input{../results/foo.tex}
  \caption{Types}
  \label{cl_spectra}
\end{figure}


\end{document}
